\documentclass{beamer}
\usepackage{verbatim}
\usepackage[T1]{fontenc}
\usetheme{Warsaw}
\title{The Rust Programming Language}
\author{Pramode C.E}
\institute {
  \texttt{http://pramode.net}
}

\begin{document}

%-----------------------

\begin{frame}[plain]
  \titlepage
\end{frame}

%---------------------
\begin{frame}{}

\begin{figure}
\includegraphics[scale=0.2]{pics/mob1.jpg}
\end{figure}

\end{frame}

%-------------------------------------
\begin{frame}{}

\begin{figure}
\includegraphics[scale=0.5]{pics/mri.jpg}
\end{figure}

\end{frame}

%-------------------------------------
\begin{frame}{}

\begin{figure}
\includegraphics[scale=0.5]{pics/jetliner.jpg}
\end{figure}

\end{frame}

%-------------------------------------
\begin{frame}{}

\begin{figure}
\includegraphics[scale=0.2]{pics/pacemaker.png}
\end{figure}

\end{frame}

%-------------------------------------
\begin{frame}{}

\begin{figure}
\includegraphics[scale=0.5]{pics/bulb.jpg}
\end{figure}

\end{frame}

%-------------------------------------
\begin{frame}{}

\begin{figure}
\includegraphics{pics/iot.jpg}
\end{figure}

\end{frame}

%-------------------------------------
\begin{frame}{}

\begin{figure}
\includegraphics[scale=0.8]{pics/car.png}
\end{figure}

\end{frame}

%-------------------------------------
\begin{frame}{}

\begin{figure}
\includegraphics[scale=0.5]{pics/wifi.png}
\end{figure}

\end{frame}

%-------------------------------------
\begin{frame}{}

\begin{figure}
\includegraphics[scale=0.3]{pics/lego.png}
\end{figure}

\end{frame}

%-------------------------------------
\begin{frame}{}

\begin{figure}
\includegraphics[scale=0.3]{pics/evm.jpg}
\end{figure}

\end{frame}

%-------------------------------------
\begin{frame}{}

\begin{figure}
\includegraphics[scale=0.8]{pics/appollo11.jpg}
\end{figure}

\end{frame}

%-------------------------------------
\begin{frame}{}

\begin{block}{}
What is common to all these?
\end{block}

\end{frame}
%---------------------------------
\begin{frame}{}

\begin{block}{}
0 1 1 0 1 1 0 1 1 0 1 1 0 0 0 1 1 0 ...
\end{block}

\end{frame}
%---------------------------------
\begin{frame}{A lot of code ...}

\begin{itemize}
\item Cardiac Pacemaker: 80,000 lines of code
\item MRI Scanner: 70 lakh lines
\item Boeing 787 dreamliner: 70 lakh lines
\item Ford GT(Car): 100 lakh lines
\item Light bulb: ??
\item Door lock: ??
\item Oven: ??
\end{itemize}

Ref: http://www.informationisbeautiful.net/visualizations/million-lines-of-code/
\end{frame}
%---------------------------------
\begin{frame}{}

\begin{figure}
\includegraphics[scale=0.2]{pics/margaret-hamilton-appollo11.jpg}
\end{figure}

\end{frame}
%---------------------------------
\begin{frame}{With code comes bugs ...}

\begin{figure}
\includegraphics[scale=0.3]{pics/bug.jpg}
\end{figure}

Code bugs can kill ... and create havoc!

\end{frame}
%---------------------------------
\begin{frame}{}

\begin{figure}
\includegraphics[scale=0.6]{pics/ritchie.jpg}
\end{figure}

\end{frame}
%---------------------------------
\begin{frame}{}

\begin{figure}
\includegraphics[scale=0.8]{pics/kr.jpg}
\end{figure}

\end{frame}
%---------------------------------
\begin{frame}{}

\begin{block}{}
What comes to your mind when you think of C?
\end{block}

\end{frame}
%---------------------------------
\begin{frame}{}

\begin{figure}
\includegraphics[scale=0.18]{pics/speed.jpg}
\end{figure}

\end{frame}
%---------------------------------
\begin{frame}{}

\begin{figure}
\includegraphics[scale=0.3]{pics/urumi.jpg}
\end{figure}

\end{frame}
%---------------------------------
\begin{frame}{C Evil Spirits}

\begin{itemize}
\item Buffer overflow / off-by-one errors
\item Null dereferencing
\item Use-after-free
\item Memory leaks
\item Dangling Pointers
\item Undefined Behaviours 
\end{itemize}

\end{frame}
%---------------------------------
\begin{frame}{C Evil Spirits}

"The second big thing each student should learn is: How can I avoid being 
burned by Cs numerous and severe shortcomings? This is a development 
environment where only the paranoid can survive"

John Regehr --- http://blog.regehr.org/archives/1393

\end{frame}
%---------------------------------
\begin{frame}{The Promise of Rust}

\begin{itemize}

\item Guaranteed Memory SAFETY without using
      Garbage Collection

\item As fast as C/C++
\item Can write OS kernels, embedded systems code
\item High level abstractions like algebraic data
      types / pattern matching with very little 
      performance overhead
\item Minimal run-time
\item Easy interfacing with C
\item Threads without data races

\end{itemize}

Ref: https://www.rust-lang.org/

\end{frame}
%----------------------------------
\begin{frame}{The Promise of Rust}

\begin{itemize}

\item Graydon Hoare started working on Rust in 2006

\item Mozilla  became interested - idea was to use it
in the experimental Servo browser engine project

\item Rust becomes an official Mozilla project

\item Several major design changes before hitting v1.0 in
May 2015. Language stabilizes

\item Regular six week release cycle

\end{itemize}

\end{frame}
%----------------------------------
\begin{frame}{Installation}

\begin{itemize}

\item Download from https://www.rust-lang.org/en-US/downloads.html

\item Untar and add folders containing "rustc" and "cargo" to the PATH

\end{itemize}

Or, you can just run:

curl -sSf https://static.rust-lang.org/rustup.sh | sh

\end{frame}
%----------------------------------
\begin{frame}{Documentation}

\begin{itemize}

\item Official "book": https://doc.rust-lang.org/book/

\item New version of the book (work-in-progress): http://rust-lang.github.io/book/

\item Rust By Example: http://rustbyexample.com/

\item O'Reilly book: http://shop.oreilly.com/product/0636920040385.do

\end{itemize}

\end{frame}

%----------------------------------
\begin{frame}{Who is using it?}

Remember, Rust is just a 1 year old baby!

\begin{itemize} 

\item Organizations running Rust in 
      Production: https://www.rust-lang.org/en-US/friends.html

\item Mozilla, for the Servo project. 

\item Dropbox story: http://www.wired.com/2016/03/epic-story-dropboxs-exodus-amazon-cloud-empire/

\item Coursera is using Rust: https://building.coursera.org/blog/2016/07/07/rust-docker-in-production-coursera/


\end{itemize}

\end{frame}

%----------------------------------
\begin{frame}{Interesting Projects}


\begin{itemize} 

\item Servo: the secure, parallel browser engine from Mozilla:https://servo.org/

\item Redox OS: Unix-like microkernel OS written in Rust: https://www.redox-os.org/

\item Tock  OS: an embedded OS for microcontrollers written in Rust: http://www.tockos.org/about/


\end{itemize}

\end{frame}

%----------------------------------
\begin{frame}{}

Rust, the most loved programming language (stackoverflow
devloper survey, 2016): 
http://stackoverflow.com/research/developer-survey-2016

\begin{figure}
\includegraphics[scale=0.2]{pics/love.png}
\end{figure}

\end{frame}
%---------------------------------
\begin{frame}{About this workshop}

This workshop is divided into 3 parts:

\begin{itemize}

\item Simple 

\item Cool  

\item Hard (and Cool!)

\end{itemize}

\end{frame}
%-------------------------------
\begin{frame}{The Simple parts}

\begin{itemize}

\item We will quickly look at: variables, control structures,
      function basics, basic types etc.

\item A demo of my first Rust project - a Raspberry Pi
      GPIO library (ported from Python)

\end{itemize}

\end{frame}

%-------------------------------
\begin{frame}{The Cool parts}

\begin{itemize}

\item Algebraic Data types

\item Pattern Matching

\item Generics and Traits

\item Iterators

\item Closures

\item Cargo
      
\end{itemize}

and some other things ...

\end{frame}
%-------------------------------

\begin{frame}{The Hard(and Cool) Parts}

This is the *core* of Rust - where the real innovation
comes in.

\begin{itemize}

\item Systems programming basics - understanding buffer overflows,
      use-after-free, memory leaks, dangling pointers etc.

\item Ownership and move semantics

\item Borrowing

\item Lifetimes

\item Unsafe Rust

      
\end{itemize}

\end{frame}
%-------------------------------

\begin{frame}{}

Let's dive into code!

\end{frame}
%-------------------------------
\begin{frame}{}

\begin{figure}
\includegraphics[scale=0.05]{pics/paper/paper1b.jpg}
\end{figure}

\end{frame}

\end{document}

